\documentclass{article}

\usepackage{amsmath}
\usepackage{diffcoeff}

\title{AMATH 383 Project}
\author{
    Alyssa Baksh, 
    Ciaran Neely,
    Gianna Biino,
    Zaheer Mohideen
}

\begin{document}

    \maketitle
    
    \hrule
    \section{Abstract}
    This is an Abstract with a citation\cite{kaveh_defining_2020}, and another citation\cite{small_movements_2021}.
    \hrule
    \section{Introduction}
    Social networks and movements have been a crucial part of society and open a window to explaining many phenomena such as herd mentality and what causes individuals to participate in social movements. \textbf{(CITE DIANI 2013)}
    
    Social movements themselves are a sociological analytical concept 'defined as networks of informal interactions between a plurality of individuals, groups and/or organizations, engaged in political or cultural conflicts, on the basis of shared collective identities.' This direction of studies has grown and evolved a lot during the last century. There are important aspects that need to be considered, when defining the dynamics of a social movement: 
    \begin{itemize}
    \item It needs to be a network based on informal interactions between groups/organisations and individuals. 
    \item It needs to contain a collective solidarity and shared beliefs.
    \item It needs to engage in conflicts, cultural or political, and promote or dispute social change.
    \item  For the majority of it, the action needs to be outside of routine and institutional procedures of social life.
    \end{itemize}
    These defining aspects, leave room for adaptation of the concept to specific examples. If for example one study focuses on a global anti-systemic social movement another will focus on a social movement supporting a local system and opposing changes to it. \textbf{(CITE DIANI 1992)}
    
    With the recent surge of social media, the studies of social movements has evolved again...
    
    it'll be fascinating to see how technology, especially in social media, plays a role in these movements in terms of their effectiveness and their methods of action. Many others have started to compare how movements founded upon social media compare to those in the past that weren't. For example, the transnational movement in Italy was analyzed to see how digital media, like Facebook, was used by activists to spread their message \cite{pavan_digital_2019}. 

    Using mathematical models, we aim to gain a deeper understanding how social movements progress with time over social media and if we can find a pattern between social movements and various parameters Specifically, we hope to find which parameters and what values are necessary for a social movement to spread? And how can they be used as tools in order to spread a message?
    It should be kept in mind, that this model will be hypothetical and will lean on simplifications in order to be simple enough to be analyzed.
    
    \section{Methods}
        Let $M(t)$ be the number of people engaged in the social movement at time $t$, and let $S(t)$ be the number of social media users who are aware of the movement at time $t$. Let $A(t)$ be the number of active social media users at time $t$, and let $P(t)$ be the population of the community where the social movement is taking place. Then, the following equations can be used:
        \begin{align}
            \diff Mt &= k_1 S(t) \left[1 - \frac{M(t)}{P(t)}\right] - k_2M(t)
            \\
            \diff St &= k_3 S(t)\left[1 - \frac{S(t)}{A(t)}\right] + k_4 F(t) - k_5 S(t)M(t)
            \\
            \diff At &= k_6 P(t) + k_7A(t) - k_8A(t)\frac{S(t)}{P(t)}
            \\
            \diff Pt &= k_9 P(t) - k_{10}M(t)
        \end{align}
        In these equations, $k_1$ is the rate at which social media users become engaged in the movement, $k_2$ is the rate at which people disengage from the movement, $k_3$ is the rate at which social media users become aware of the movement, $k_4$ is the rate at which influencers promote the movement, $k_5$ is the rate at which people disengage from the movement due to social media overload, $F(t)$ is the number of trending posts related to the movement at time $t$, $k_6$ is the rate at which the population of the community where the movement is taking place increases, $k_7$ is the rate at which social media users join the platform, and $k_8$ is the rate at which social media users leave the platform due to social media overload or other factors.
        
        The first equation describes how the size of the movement changes over time, with the number of people becoming engaged in the movement being proportional to the number of social media users who are aware of it and the available population who may join the movement, but also subject to a rate of disengagement.
        
        The second equation describes how the awareness of the movement changes over time, with the number of social media users becoming aware of it being proportional to the number of users who are already aware of it, as well as the rate at which influencers promote it. The rate of awareness is also subject to a rate of disengagement due to social media overload, which is proportional to the product of the number of social media users and the number of people engaged in the movement.
        
        The third equation describes how the number of active social media users changes over time, with the number of users increasing due to population growth and social media adoption, but also subject to a rate of decline due to social media overload or other factors, which is proportional to the product of the number of social media users and the number of people engaged in the movement relative to the population size.
        
        These equations can be used to model the growth and spread of a social movement on social media while accounting for additional factors such as population growth, social media adoption, and social media overload. The parameters can be adjusted to reflect the specific characteristics of the movement and the social media platforms being used.

        In this equation, k9 is the rate at which the population of the community where the social movement is taking place increases due to factors such as birth and immigration. The rate of population growth is subject to a rate of decline due to people disengaging from the movement, which is proportional to the number of people engaged in the movement.

        This equation captures the influence of the community's population size on the social movement and how the movement may affect the community's population growth over time.
    \section{Results}
    \section{Discussion}
    \newpage
    \bibliographystyle{plain}
    \bibliography{references}
    

\end{document}